\documentclass[12pt]{article}

\usepackage[english]{babel}
\usepackage[utf8x]{inputenc}
\usepackage{amsmath}
\usepackage{enumitem}
\usepackage{graphicx}
\usepackage{ulem}
\usepackage{caption}
\usepackage{placeins}
\usepackage[usenames,dvipsnames]{color}
\usepackage[colorinlistoftodos]{todonotes}
\usepackage{listings}
\usepackage{fixltx2e}
\usepackage{scrpage2}
\usepackage{lastpage}
\clearscrheadfoot
\pagestyle{scrheadings}
\usepackage{glossaries}
\usepackage[
top    = 2.75cm,
bottom = 2.00cm,
left   = 2.50cm,
right  = 2.00cm]{geometry}
\setcounter{secnumdepth}{4}


\makeglossaries

\newglossaryentry{erp} {name=ERP, description={Enterprise Resource Planning}}
\newglossaryentry{glossaryVerweis} {name=abkuerzung, description={Langer Name}}


\begin{document}
\begin{titlepage}
\begin{center}
% Oberer Teil der Titelseite:
\includegraphics[width=0.9\textwidth]{images/vwlogo}\\[1cm]    


% Title
\rule{1.0\textwidth}{1mm}
{ \huge \bfseries \\[0.4cm]  \huge ERP-Evaluation \\[0.4cm] }
\LARGE TGM - HTBLuVA Wien XX \\ IT Department  \\[0.4cm]

\rule{1.0\textwidth}{1mm}




% Author and supervisor
\noindent 
\vspace{3cm}

\begin{center}
\large
Authors: \\ 
Bergler \textsc{Adrian} \&
Haidn \textsc{Martin} \&
Siegel \textsc{Hannah} \&
Soyka \textsc{Wolfram}
\end{center}

\vfill

% Bottom of the page
{\large \today}

\end{center}
\end{titlepage}

\tableofcontents


%HEADER AND FOOTER
\pagenumbering{arabic}
\ohead{\headmark}
\automark{section}
\ifoot{© Bergler, Haidn, Siegel, Soyka}
\ofoot{\pagemark ~of \pageref{LastPage}}

\newpage

\section{Auswahl des Unternehmens}

\section{Volkswagen AG}
http://www.economist.com/node/21558269
\subsection{Das Unternehmen}
\subsubsection{Historie}

\subsection{Finanzen}
%  3 u 4
\textbf{Umsatz}\\
Im Jahr 2013 lag der Umsatzerlös bei 65.587 Millionen Euro. Im Jahr zuvor jedoch betrug dieser 68.361 Millionen Euro. Ein Rückgang von 226 Millionen Euro ist dadurch entstanden. Gleichzeitig ist auch ein Gewinnrückgang von mehr als 3 Milliarden Euro festzuhalten. Gründe dafür lassen sich wie folgt finden:
\\
"Die in Vorjahren erworbenen 73,7\% der Anteile am Grundkapital der MAN SE, München, (9,1 Mrd.€) wurden von
der Volkswagen AG im Geschäftsjahr in die Truck \& Bus GmbH, eine 100-prozentige Tochtergesellschaft eingebracht.
Zusätzlich hat die Volkswagen AG 3,3 Mrd.€ in die Kapitalrücklage der Truck \& Bus GmbH eingezahlt. Von der Truck \&
Bus GmbH wurden 2013 insgesamt 1,0 Mrd. € Verluste aufgrund des Beherrschungs- und Gewinnabführungsvertrags
mit der MAN SE übernommen. ",\cite[Seite 3]{jbilanz2013vw}
\\\\
"Die Volkswagen AG hat von der Volkswagen Bank GmbH, Braunschweig, eine Beteiligung erworben und diese anschließend im Wege der Sacheinlage (1,7 Mrd.€) in die VW Finance Luxemburg S.A., Luxemburg, eingebracht.\\
Darüber hinaus wurden Kapitalzuführungen bei der AUDI AG, Ingolstadt, (1,9 Mrd.€) und kleinere Kapitalmaß-
nahmen bei verbundenen Unternehmen durchgeführt. Bei der der Global Automotive C.V. Amsterdam, Niederlande
wurde eine Sachkapitalherabsetzung. (1,1 Mrd.€) durchgeführt. Die Volkswagen AG hat im HI-TV Fonds (TreasuryFonds) 1,0 Mrd.€ angelegt. ",\cite[Seite 4]{jbilanz2013vw}
\\ \\ 
\textbf{Aktie} \\
"On 7 April 1961, the Volkswagen Share was traded for the first time on a regulated open market, thus writing a chapter of economic history.",\cite{vsc50y} \\
\\
Der Stand am Mittwoch, dem 01. April 2015 der VW Vorzugsaktie (Xetra) war bei 245,50.

Marktkap. total (Stamm + Vorzug): EUR 114,71 Mrd.



\begin{figure}[here!]
\centering
\includegraphics[width=0.7\textwidth]{images/finanzen2015}
\caption{Zehn Jahres Übersicht der VW-Vorzugsaktie \cite{aktienfotos}}
\end{figure}\FloatBarrier
\noindent
Die VM Vorzugsaktie ist in den letzten zehn Jahren stetig gestiegen. In 2008 wurde der Fall der Aktie durch die Finanzkrise bedingt.
The ten year evolution of the stock shows a very small upward trend, after the financial crisis in 2008 there was a drop in the stock's value but it has been improving constantly since then. 
\begin{figure}[here!]
\centering
\includegraphics[width=0.7\textwidth]{images/finanzen2015S}
\caption{Zehn Jahres Übersicht der VW-Stammaktie \cite{aktienfotos}}
\end{figure}\FloatBarrier
\noindent
%TODO mama fragen
%(Reuters) - Volkswagen (VOWG.DE) briefly became the world's biggest company by market value on Tuesday, as short sellers caught betting on a price drop with borrowed stock scrambled to find shares after a buying spree by Porsche (PSHG_p.DE).Short sellers desperate to close their positions paid as much as 1,005 euros a share during the session following Sunday's news that there was less than 6 percent of VW voting stock still floating in the market.At that price Volkswagen's voting stock was worth 296 billion euros ($370 billion), or more than the $343 billion market capitalization of Exxon Mobil (XOM.N). VW shares later closed trading on Tuesday up 82 percent at 945 euros.
% % % \cite{2008wtf} % % %

\begin{figure}[here!]
\centering
\includegraphics[width=0.7\textwidth]{images/finanzen20151}
\caption{Jahres Übersicht der VW-Aktie \cite{aktienfotos}}
\end{figure}\FloatBarrier
\noindent
\textbf{Aktinonärsstruktur}\\
Mit 31.12.2014 waren insgesammt 180.641.478 Vorzugsaktien und 295.089.818 Stammaktien ausstehend.\\
Im Juni 2014 hat die Volkswagen Aktiengesellschaft 10.471.204 neue Vorzugsaktien ausgegeben. Zusätzlich wurden im 1. Halbjahr 2014 22.103 Vorzugsaktien aus der Wandlung von Pflichtwandelanleihen geschaffen. Zum Stichtag 30. Juni 2014 setzte sich das gezeichnete Kapital der Volkswagen Aktiengesellschaft aus 295.089.818 Stammaktien und 180.641.478 Vorzugsaktien zusammen.
\cite{aktionaersstruktur} \\ \\
\textit{Stimmrechtsverteilung} \\
50,73\% Porsche Automobil Holding SE, Stuttgart\\
20,00\% Land Niedersachsen, Hannover\\
17,00\% Qatar Holding LLC\\
12,30\% Weitere
\\ \\
\textbf{Recovery}\\
"When Ferdinand Piëch arrived as Volkswagen's chief executive in 1993, things looked dire. The carmaker was overspending, overmanned and inefficient, and had lost its reputation for quality. How things have changed: last year the VW group's profits more than doubled, to a record €18.9 billion (\$23.8 billion). As other European volume carmakers seek to close factories and cut jobs, VW is seizing market share in Europe, booming in China and staging a comeback in America. It plans to spend €76 billion on new models and new factories by 2016. Its global workforce is more than half a million, and growing."\cite{ec2}
\\ \\
\textbf{Sales in America} \\
"The firm reported some of its best sales figures since the era of the original Beetle—and said it wanted to sell at least 800,000 vehicles per year in America by 2018.\\
Then, rather suddenly, things went south. Although Volkswagen of America is still well ahead of where it was before the Great Recession, it has suffered two consecutive years of declining sales. And in the first five months of 2014, as rivals such as GM posted some of their best numbers in a decade, the VW brand’s sales dropped by another 15\%. With sales in America barely above 400,000 in 2013, down 7\% from the previous year, doubts have been growing as to whether VW will be able to reach its target of 800,000 by 2018." \cite{ec1}

\newpage
\subsection{Produktspektrum}
Im Mittelpunkt der Produktion steht das Automobil und wird von eine Anzahl an vielseitigen Dienstleistungen rund um das Thema Fahren verstärkt.
Das Produktspektrum der Volkswagen AG beinhaltet alle Kfz vom Stadtfahrzeug, über Motorad, bis zum Großtransporter.
So unterteilt sich der Konzern in die Folgenden Marken und Tochtergesellschafen.
\begin{figure}[here!]
\centering
\includegraphics[width=0.7\textwidth]{images/Volkswagen-Group-Brands}
\caption{Marken und Tochtergesellschaften der Volkswagen AG. \cite{marken}}
\end{figure}\FloatBarrier
\noindent
\textbf{Volkswagen Group}\\
Das Segment der Volkswagen Group ist überwiegend auf den Endverbaucher Abgestimmt. Es beinhaltet den kleinen Stadtflitzer bis zum SUV. Die klingensten Modelle darunter sind unter anderem Käfer, Polo, Golf, Passat und Sharan.
Auch im elektrifizierten Modellbereich ist die Marke mit der e-up! und e-golf Serie vertreten.\\
\\
\textbf{Audi}\\
Audi produziert seither in den gängigen Klassen A, S, T, Q und weitere Kleinserien in der typischen Buchstabenbenennung. Seit mittlerweile 60 Jahren der Produktion sind die Klassen in ihrer achten Generation angelangt und bietet seinen Fahrern eine weitgestreutet Modellpallette von Kleinwagen, über Sportwagen und SUV, bis zur Oberkalsse.\\
\\
\textbf{Skoda}\\
Die Modelle der Marke Skoda sind eher in der Mittelklasse orientiert, bieten dem Fahrer allerdings zu günstigeren Preisen als manche andere Konzerntöchter, eine Vielzahl von aktuellen Features und neuen technologien zum Fahrkomfort und Sicherheit.
Die gängisten Modelle sind Fabia, Rapid und Octavia.

\newpage
\textbf{Seat}\\
Der Spanische Marke Seat ist im Kleinwagen und Mittelklasseberreich zu Hause. Der überwiegende Teil der produzierten Serien sind Lizenzbauten der Marke Fiat und den Volkswagen eigenen Marken VW, Audi und Skoda.
Die gängigsten Serien der Marke sind unter anderen Ibiza und Leon.\\
\\ 
\textbf{Porsche}\\
Die Luksusmarke Porsche entwickelt ausschließlich Sportwagen, Oberklassenmodelle und SUV's.
Die gängisten Serien des Herstellers sind Cayenne, Boxter und 911, von denen Carriera die größte unterserie besitzt.\\
\\
\textbf{Lamborghini}\\
Der Luxushersteller produziert hauptsächlich Sportwagen und Coupes in Serie. Neben den Serienmodellen wurden auch eine
Vielzahl von Einzelmodellen entworfen, die in Design und Ergonomität, dafür aber auch Preis glänzen.\\
\\
\textbf{Bentley}\\
Der Hoflieferat für die brittische Königsfamilie ist ebenfalls für seine Modelle in der Ober- und Sportklasse bekannt.

\textbf{Bukatti}\\
\textbf{Ducati}\\
\textbf{Scania}\\
\textbf{MAN}


\newpage
\subsection{Standorte und Unternehmensstruktur}
Der Konzern betreibt über 118 Fertigungsstätten in denen rund 600.000 Personen beschäftigt sind.
Der überwiegende Teil ist nach Afrika und Asien ausgelagert, der rest wird zum Teil in Amerika und Europa in den
folgenden Ländern Produziert. \cite{produktionsstandorte}
\begin{table}[h]
	\begin{tabular}{|l|l|l|l|l|}
		\hline
		Argentinien          & Bosnien und Herzegovina & Brasilien & China    & Deutschland          \\ \hline
		Indien               & Mexiko                  & Polen     & Portugal & Russische Föderation \\ \hline
		Slowakische Republik & Spanien                 & Südafrika & USA      &                      \\ \hline
	\end{tabular}
\end{table}
\\\\
"Die Volkswagen AG stellt die Muttergesellschaft des eigentlichen Volkswagen Konzerns da und entwickelt insbesondere Pkw und Nutzfahrzeuge für den Vertrieb, sowie Fahrzeuge und deren Komponenten für den Konzern.
"Der Vorstand der Volkswagen AG leitet das Unternehmen in eigener Verantwortung. Der Aufsichtsrat bestellt, überwacht und berät den Vorstand und ist in Entscheidungen, die von grundlegender Bedeutung für das Unternehmen sind, unmittelbar eingebunden." \cite{struktur}



\subsection{Logistik}
\subsection{Aufbau- und Ablauforganisation}
\subsection{Markt}
%Marktumfeld, -situation, -wert, Trends

\subsection{Ziele und Zukunftsaspekte}

"Volkswagen has its eye on emerging markets with the new e-Golf, a version of its ever-popular hatchback line, as well as the smaller e-Up! “The empire strikes back,” proclaimed Heinz-Jakob Neusser, as the two models rolled onto the stage at a launch event during the media-preview days in Frankfurt. Europe’s largest carmaker now says it wants to have 40 hybrids, plug-ins and BEVs in the showrooms of its dozen brands by 2018. This represents a big shift: at the beginning of the decade the firm’s senior executives were still dismissing battery technology.",\cite{ec3}

"VOLKSWAGEN is nothing if not ambitious: its plan is to dethrone Toyota as the world’s biggest carmaker. It has already nosed past rival General Motors and has plenty of momentum in China, the world’s largest car market. But one key country is holding back the German giant: America.", \cite{ec1}
\section{Auswahl eines ERP Systemes}
\section*{Projekthandbuch}
\section*{Pflichtenheft}
\section*{Vorbereitung Kick Off Meeting}




\newpage
\listoftables
\listoffigures
\printglossaries
\subsection{Easy Bibliography}
\begin{thebibliography}{56}

\bibitem{name}
%TODO
   \textbf{Who}, When\\
  \textit{url}
  \newline last used: dd.mm.yyyy, hh:mm
 
 \bibitem{ec1} 
  \textbf{Beetling back to success}, Jun 24th 2014, P.E, The Economist\\
  \textit{http://www.economist.com/blogs/schumpeter/2014/06/volkswagen-america}
  \newline last used: 07.03.2015, 13:00
  
   
 \bibitem{ec2} 
  \textbf{VW conquers the world - Germany’s biggest carmaker is leaving rivals in the dust}\\, Jul 7th 2012 , The Economist\\
  \textit{  http://www.economist.com/node/21558269}
  \newline last used: 07.03.2015, 13:07
  
    
   
 \bibitem{ec3} 
  \textbf{Europe goes electric - The Frankfurt motor show
}\\,Sep 12th 2013 , P.E., The Economist\\
  \textit{  http://www.economist.com/node/21558269}
  \newline last used: 07.03.2015, 13:09
  
   \bibitem{vsc50y} 
  \textbf{Volkswagen Share celebrates its 50th birthday}\\, Jun 4th 2011 , Volkswagen AG\\
  \textit{http://www.volkswagenag.com/content/vwcorp/info\_center/en/themes/2011/04/Volkswagen\_Share\_celebrates\_its\_50th\_birthday.html
}
  \newline last used: 07.03.2015, 13:16
  
   \bibitem{2008kurs} 
  \textbf{Volkswagen Share celebrates its 50th birthday}\\, Jun 4th 2011 , Volkswagen AG\\
  \textit{  http://www.boerse.de/boersenwissen/boersengeschichte/Kurskapriolen-der-VW-Aktie-2008-\%7C45}
  \newline last used: 07.03.2015, 13:24
  

   \bibitem{jbilanz2013vw} 
  \textbf{ABSCHLUSS VOLKSWAGEN AG }, 2013 \\
  \textit{ http://www.volkswagenag.com/content/vwcorp/info\_center/de/publications/2014/03/Financial\_Statements\_VWAG\_2013.bin.html/binarystorageitem/file/Abschluss+Volkswagen+AG+2013\_deutsch.pdf  
}
  \newline last used: 08.03.2015, 14:03
    
    
       \bibitem{aktionaersstruktur} 
  \textbf{Aktionärsstruktur Volkswagen AG}, Stand 31.12.2014 \\
  \textit{http://www.volkswagenag.com/content/vwcorp/content/de/investor\_relations/share/Shareholder\_Structure.html}
  \newline last used: 01.04.2015, 15:04
  
       \bibitem{2008wtf} 
 \textbf{Short sellers make VW the world's priciest firm}, reuters.com. SARAH Marsh \\
  \textit{  http://www.reuters.com/article/2008/10/28/us-volkswagen-idUSTRE49R3I920081028}
  \newline last used: 01.04.2015, 15:39  
  
         \bibitem{aktienfotos} 
 \textbf{Vorzugs- und Stammaktien}, volkswagenag.com \\
  \textit{   http://www.volkswagenag.com/content/vwcorp/content/de/investor\_relations/share.html}
  \newline last used: 01.04.2015, 15:57
  
  \bibitem{marken} 
 \textbf{Web Ressource}, marketbusinessnews.com \\
  \textit{   http://marketbusinessnews.com/wp-content/uploads/2014/02/Volkswagen-Group-Brands.png}
  \newline last used: 01.04.2015, 16:33
  
  \bibitem{produktionsstandorte} 
 \textbf{Produktionsstandorte}, mvolkswagenag.com \\
  \textit{   http://www.volkswagenag.com/content/vwcorp/content/de/the\_group/production\_plants.html}
  \newline last used: 01.04.2015, 17:02
  
  \bibitem{struktur} 
 \textbf{Struktur und Geschäftstätigkeit}, volkswagenag.com \\
  \textit{	http://www.volkswagenag.com/content/gb2007/content/de/corporate\_governance/structure\_and\_business\_activities\_\_part\_of\_the\_management\_report\_.html}
  \newline last used: 01.04.2015, 17:02
  
\end{thebibliography}
\end{document}
