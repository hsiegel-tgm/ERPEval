\documentclass[12pt]{article}

\usepackage[english]{babel}
\usepackage[utf8x]{inputenc}
\usepackage{amsmath}
\usepackage{enumitem}
\usepackage{graphicx}
\usepackage{ulem}
\usepackage{caption}
\usepackage{placeins}
\usepackage[usenames,dvipsnames]{color}
\usepackage[colorinlistoftodos]{todonotes}
\usepackage{listings}
\usepackage{fixltx2e}
\usepackage{scrpage2}
\usepackage{lastpage}
\clearscrheadfoot
\pagestyle{scrheadings}
\usepackage{glossaries}
\usepackage[
top    = 2.75cm,
bottom = 2.00cm,
left   = 2.50cm,
right  = 2.00cm]{geometry}
\setcounter{secnumdepth}{4}


\makeglossaries

\newglossaryentry{erp} {name=ERP, description={Enterprise Resource Planning}}
\newglossaryentry{glossaryVerweis} {name=abkuerzung, description={Langer Name}}


\begin{document}
\begin{titlepage}
\begin{center}
% Oberer Teil der Titelseite:
\includegraphics[width=0.5\textwidth]{images/logo}\\[1cm]    

\LARGE TGM - HTBLuVA Wien XX \\ IT Department  \\[1.5cm]

% Title
\rule{1.0\textwidth}{1mm}
{ \huge \bfseries \\[0.4cm]  \huge Volkswagen AG \\ \LARGE ERP-Evaluation \\[0.4cm] }

\rule{1.0\textwidth}{1mm}



% Author and supervisor
\noindent 
\vspace{5cm}

\begin{center}
\large
Authors: \\ 
Bergler \textsc{Adrian} \&
Haidn \textsc{Martin} \&
Siegel \textsc{Hannah} \&
Soyka \textsc{Wolfram}
\end{center}

\vfill

% Bottom of the page
{\large \today}

\end{center}
\end{titlepage}

\tableofcontents


%HEADER AND FOOTER
\pagenumbering{arabic}
\ohead{\headmark}
\automark{section}
\ifoot{© Haidn, Siegel, Soyka}
\ofoot{\pagemark ~of \pageref{LastPage}}

\newpage

\section{Auswahl des Unternehmens}

\section{Volkswagen AG}
http://www.economist.com/node/21558269
\subsection{Das Unternehmen}
\subsubsection{Historie}

\subsection{Finanzen}

\textbf{Stock}
"On 7 April 1961, the Volkswagen Share was traded for the first time on a regulated open market, thus writing a chapter of economic history.",\cite{vsc50y} \\




\textbf{Recovery}
"When Ferdinand Piëch arrived as Volkswagen's chief executive in 1993, things looked dire. The carmaker was overspending, overmanned and inefficient, and had lost its reputation for quality. How things have changed: last year the VW group's profits more than doubled, to a record €18.9 billion (\$23.8 billion). As other European volume carmakers seek to close factories and cut jobs, VW is seizing market share in Europe, booming in China and staging a comeback in America. It plans to spend €76 billion on new models and new factories by 2016. Its global workforce is more than half a million, and growing."\cite{ec2}
\\ \\
\textbf{Sales in America} \\
"The firm reported some of its best sales figures since the era of the original Beetle—and said it wanted to sell at least 800,000 vehicles per year in America by 2018.\\
Then, rather suddenly, things went south. Although Volkswagen of America is still well ahead of where it was before the Great Recession, it has suffered two consecutive years of declining sales. And in the first five months of 2014, as rivals such as GM posted some of their best numbers in a decade, the VW brand’s sales dropped by another 15\%. With sales in America barely above 400,000 in 2013, down 7\% from the previous year, doubts have been growing as to whether VW will be able to reach its target of 800,000 by 2018." \cite{ec1}

\subsection{Produktspektrum}
\subsection{Standorte und Unternehmensstruktur}

\subsection{Logistik}
\subsection{Aufbau- und Ablauforganisation}
\subsection{Markt}
%Marktumfeld, -situation, -wert, Trends

\subsection{Ziele und Zukunftsaspekte}

"Volkswagen has its eye on emerging markets with the new e-Golf, a version of its ever-popular hatchback line, as well as the smaller e-Up! “The empire strikes back,” proclaimed Heinz-Jakob Neusser, as the two models rolled onto the stage at a launch event during the media-preview days in Frankfurt. Europe’s largest carmaker now says it wants to have 40 hybrids, plug-ins and BEVs in the showrooms of its dozen brands by 2018. This represents a big shift: at the beginning of the decade the firm’s senior executives were still dismissing battery technology.",\cite{ec3}

"VOLKSWAGEN is nothing if not ambitious: its plan is to dethrone Toyota as the world’s biggest carmaker. It has already nosed past rival General Motors and has plenty of momentum in China, the world’s largest car market. But one key country is holding back the German giant: America.", \cite{ec1}
\section{Auswahl eines ERP Systemes}
\section{Projekthandbuch}
\section{Pflichtenheft}
\section{Vorbereitung Kick Off Meeting}




\newpage
\listoftables
\listoffigures
\printglossaries
\subsection{Easy Bibliography}
\begin{thebibliography}{56}

\bibitem{name}
%TODO
   \textbf{Who}, When\\
  \textit{url}
  \newline last used: dd.mm.yyyy, hh:mm
 
 \bibitem{ec1} 
  \textbf{Beetling back to success}, Jun 24th 2014, P.E, The Economist\\
  \textit{http://www.economist.com/blogs/schumpeter/2014/06/volkswagen-america}
  \newline last used: 07.03.2015, 13:00
  
   
 \bibitem{ec2} 
  \textbf{VW conquers the world - Germany’s biggest carmaker is leaving rivals in the dust}\\, Jul 7th 2012 , The Economist\\
  \textit{  http://www.economist.com/node/21558269}
  \newline last used: 07.03.2015, 13:07
  
    
   
 \bibitem{ec3} 
  \textbf{Europe goes electric - The Frankfurt motor show
}\\,Sep 12th 2013 , P.E., The Economist\\
  \textit{  http://www.economist.com/node/21558269}
  \newline last used: 07.03.2015, 13:09
  
   \bibitem{vsc50y} 
  \textbf{Volkswagen Share celebrates its 50th birthday}\\, Jun 4th 2011 , Volkswagen AG\\
  \textit{http://www.volkswagenag.com/content/vwcorp/info_center/en/themes/2011/04/Volkswagen\_Share\_celebrates\_its\_50th\_birthday.html
}
  \newline last used: 07.03.2015, 13:16
  
   \bibitem{2008kurs} 
  \textbf{Volkswagen Share celebrates its 50th birthday}\\, Jun 4th 2011 , Volkswagen AG\\
  \textit{  http://www.boerse.de/boersenwissen/boersengeschichte/Kurskapriolen-der-VW-Aktie-2008-\%7C45}
  \newline last used: 07.03.2015, 13:24
  
  
\end{thebibliography}
\end{document}
